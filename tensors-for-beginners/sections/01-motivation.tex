\section{Motivation: Coordinate Independence in Physics}

Physical laws must not depend on how we label space.

\subsection*{Why Does This Matter?}

In physics, we describe phenomena using mathematical equations. However, these equations should describe reality itself, not our particular choice of coordinates. Consider:

\begin{itemize}
    \item \textbf{Newton's Laws}: The force on an object doesn't change if you rotate your coordinate system
    \item \textbf{Maxwell's Equations}: Electromagnetic fields behave the same regardless of your reference frame
    \item \textbf{General Relativity}: The curvature of spacetime is independent of coordinate choice
\end{itemize}

This principle is fundamental: \emph{physical reality is invariant, even though our descriptions of it may change with coordinates}.

\subsection*{Example: Velocity}
A particle moving in space has a velocity that exists independently of coordinates.
If we rotate our coordinate axes, the particle does not suddenly move differently — only the \emph{numbers} describing its velocity change.

For instance, a ball moving northward at 5 m/s has the same physical motion whether you describe it in:
\begin{itemize}
    \item Coordinates aligned north-south and east-west
    \item Coordinates rotated 45 degrees
    \item Polar coordinates centered at your location
\end{itemize}

The velocity vector itself (the geometric object) remains the same. Only its components (the numbers in your coordinate system) change.

\subsection*{The Core Distinction}

This distinction between:
\begin{itemize}
    \item the geometric object (velocity)
    \item its coordinate representation (components)
\end{itemize}
is the core motivation for tensors.

\subsection*{Key Principle}
\begin{quote}
If a quantity represents something physical or geometric, then changing coordinates must not change the object itself, only how we describe it.
\end{quote}

Tensors are defined so this principle is automatically satisfied. They provide a mathematical framework where:
\begin{enumerate}
    \item Objects exist independently of coordinates
    \item We can compute how their components transform when we change coordinates
    \item Physical laws written in tensor notation are automatically coordinate-independent
\end{enumerate}

