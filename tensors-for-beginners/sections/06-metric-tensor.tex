\section{Metric Tensor}

We start this chapter by asking a supposedly simple question, how do you get the length of a vector?
So let's bring back our usual diagram with a sample vector living in our 2D space with a couple of basis vectors defined.

\input{figures/06-vector-length-example.tex}

\begin{align*}
    \begingroup\color{orange}\vec{v}\endgroup &= 1\begingroup\color{blue}\vec{e_1}\endgroup + 2\begingroup\color{blue}\vec{e_2}\endgroup\\
    &= 5/4\begingroup\color{red}\tilde{\vec{e_1}}\endgroup + 3\begingroup\color{red}\tilde{\vec{e_2}}\endgroup
\end{align*}

Obviously, we can use Pythagora's theorem in the $\color{blue}\vec{e_i}$ basis, and compute 
the length of the vector as:

\begin{align*}
    \|\begingroup\color{orange}{\vec{v}}\endgroup\|^2 &= (1)^2 + (2)^2 = 5\\
    \|\begingroup\color{orange}{\vec{v}}\endgroup\| &= \sqrt{5} \approx 2.236
\end{align*}

Well let's try doing the same in the new basis $\color{red}\tilde{\vec{e_i}}$, why not?

\begin{align*}
    \|\begingroup\color{orange}{\vec{v}}\endgroup\|^2 &= (5/4)^2 + (3)^2 = 169/16\\
    \|\begingroup\color{orange}{\vec{v}}\endgroup\| &= \sqrt{169/16} \approx 3.24
\end{align*}

Mmm, well that does not seems right, doesn't it?
\textbf{Reason is that Pythagora's theorem only works in orthonormal basis like $\color{blue}\vec{e_i}$, and it does
not yield the correct result when used in a non-orthonormal one like $\color{red}\tilde{\vec{e_i}}$}


So what's the correct formula to use to get the length of a vector, independently of the basis used?
This is the \textbf{dot product formula}, which in the two different coordinate systems (i.e. basis vectors):

\begin{align*}
    \|\begingroup\color{orange}\vec{v}\endgroup\|^2 &= \begingroup\color{orange}\vec{v}\endgroup \cdot \begingroup\color{orange}\vec{v}\endgroup\\
    &= \left(\begingroup\color{cyan}v^1\endgroup \begingroup\color{blue}\vec{e_1}\endgroup + \begingroup\color{cyan}v^2\endgroup \begingroup\color{blue}\vec{e_2}\endgroup\right)
    \cdot \left(\begingroup\color{cyan}v^1\endgroup \begingroup\color{blue}\vec{e_1}\endgroup + \begingroup\color{cyan}v^2\endgroup \begingroup\color{blue}\vec{e_2}\endgroup\right)\\
    &= \bigl(\begingroup\color{cyan}v^1\endgroup\bigr)^2\left(\begingroup\color{blue}\vec{e_1}\endgroup\cdot\begingroup\color{blue}\vec{e_1}\endgroup\right)
    + 2 \begingroup\color{cyan}v^1v^2\endgroup\left(\begingroup\color{blue}\vec{e_1}\endgroup\cdot\begingroup\color{blue}\vec{e_2}\endgroup\right)
    + \bigl(\begingroup\color{cyan}v^2\endgroup\bigr)^2\left(\begingroup\color{blue}\vec{e_2}\endgroup\cdot\begingroup\color{blue}\vec{e_2}\endgroup\right) \\ \\
    &= \bigl(\begingroup\color{magenta}\tilde{v^1}\endgroup\bigr)^2\left(\begingroup\color{red}\tilde{\vec{e_1}}\endgroup\cdot\begingroup\color{red}\tilde{\vec{e_1}}\endgroup\right)
    + 2 \begingroup\color{magenta}\tilde{v^1v^2}\endgroup\left(\begingroup\color{red}\tilde{\vec{e_1}}\endgroup\cdot\begingroup\color{red}\tilde{\vec{e_2}}\endgroup\right)
    + \bigl(\begingroup\color{magenta}\tilde{v^2}\endgroup\bigr)^2\left(\begingroup\color{red}\tilde{\vec{e_2}}\endgroup\cdot\begingroup\color{red}\tilde{\vec{e_2}}\endgroup\right)
\end{align*}

If we try applying this, in the orthonormal basis $\color{blue}\vec{e_i}$, it will simplify to the Pythagora's Theorem, whereas applying this
to the new basis $\color{red}\tilde{\vec{e_i}}$, you will see that, the coefficients obtained are not simply 1s and 0s, but are 
different numbers, that eventually allow the result to be consistent, and the length to be exactly the same computed in the old basis.\\

If you try it yourself, using the forward transform and getting the new basis in terms of the old, and computing the dot products, you will find that:

\begin{align*}
    \begingroup\color{red}\tilde{\vec{e_1}}\endgroup\cdot\begingroup\color{red}\tilde{\vec{e_1}}\endgroup &= 5\\
    \begingroup\color{red}\tilde{\vec{e_1}}\endgroup\cdot\begingroup\color{red}\tilde{\vec{e_2}}\endgroup &= -3/4\\
    \begingroup\color{red}\tilde{\vec{e_2}}\endgroup\cdot\begingroup\color{red}\tilde{\vec{e_2}}\endgroup &= 5/16
\end{align*}

Now, the formula for the length using the dot product above, can be rewritten in matrix terms as:

\begin{align*}
    \|\begingroup\color{orange}\vec{v}\endgroup\|^2 &= \begingroup\color{orange}\vec{v}\endgroup \cdot \begingroup\color{orange}\vec{v}\endgroup\\
    &= \bigl(\begingroup\color{cyan}v^1\endgroup\bigr)^2
    + \bigl(\begingroup\color{cyan}v^2\endgroup\bigr)^2\\ 
    &= 5 \bigl(\begingroup\color{magenta}\tilde{v^1}\endgroup\bigr)^2
    + 2(-3/4) \begingroup\color{magenta}\tilde{v^1}\endgroup\begingroup\color{magenta}\tilde{v^2}\endgroup
    + (5/16) \bigl(\begingroup\color{magenta}\tilde{v^2}\endgroup\bigr)^2\\ \\
    &= \begin{bmatrix}
        \begingroup\color{cyan}v^1\endgroup & \begingroup\color{cyan}v^2\endgroup
    \end{bmatrix} 
    \begin{bmatrix}
        1 & 0\\
        0 & 1
    \end{bmatrix} 
    \begin{bmatrix}
        \begingroup\color{cyan}v^1\endgroup\\
        \begingroup\color{cyan}v^2\endgroup 
    \end{bmatrix}\\
    &= \begin{bmatrix}
        \begingroup\color{magenta}\tilde{v^1}\endgroup & \begingroup\color{magenta}\tilde{v^2}\endgroup
    \end{bmatrix} 
    \begin{bmatrix}
        5 & -3/4\\
        -3/4 & 5/16
    \end{bmatrix} 
    \begin{bmatrix}
        \begingroup\color{magenta}\tilde{v^1}\endgroup\\
        \begingroup\color{magenta}\tilde{v^2}\endgroup 
    \end{bmatrix}
\end{align*}

So as you see, the key to getting the correct length, in any coordinate system, is that matrix in between you see there.
\textbf{And that is what we call the metric tensor}

\begin{align*}
    g_{\color{blue}\vec{e_i}} = \begin{bmatrix}
        1 & 0\\
        0 & 1
    \end{bmatrix} &&
    g_{\color{red}\tilde{\vec{e_i}}} = \begin{bmatrix}
        5 & -3/4\\
        -3/4 & 5/16
    \end{bmatrix}
\end{align*}

This is a tensor, as it's invariant, but its components are different in different coordinate systems.
And if you notice, at the end of the day, the components of the metric tensor in a specific basis, are the dot products between the basis vectors.

\begin{align*}
    g_{\color{blue}\vec{e_i}} = \begin{bmatrix}
        \begingroup\color{blue}\vec{e_1}\endgroup\cdot\begingroup\color{blue}\vec{e_1}\endgroup & \begingroup\color{blue}\vec{e_1}\endgroup\cdot\begingroup\color{blue}\vec{e_2}\endgroup\\
        \begingroup\color{blue}\vec{e_2}\endgroup\cdot\begingroup\color{blue}\vec{e_1}\endgroup & \begingroup\color{blue}\vec{e_2}\endgroup\cdot\begingroup\color{blue}\vec{e_2}\endgroup
    \end{bmatrix} &&
    g_{\color{red}\tilde{\vec{e_i}}} = \begin{bmatrix}
        \begingroup\color{red}\tilde{\vec{e_1}}\endgroup\cdot\begingroup\color{red}\tilde{\vec{e_1}}\endgroup & \begingroup\color{red}\tilde{\vec{e_1}}\endgroup\cdot\begingroup\color{red}\tilde{\vec{e_2}}\endgroup\\
        \begingroup\color{red}\tilde{\vec{e_2}}\endgroup\cdot\begingroup\color{red}\tilde{\vec{e_1}}\endgroup & \begingroup\color{red}\tilde{\vec{e_2}}\endgroup\cdot\begingroup\color{red}\tilde{\vec{e_2}}\endgroup
    \end{bmatrix}
\end{align*}

So to sum up, and using the Einstein summation convention:

\begin{align*}
    \|\begingroup\color{orange}\vec{v}\endgroup\|^2 &= \begingroup\color{cyan}v^i v^j \endgroup \left(\begingroup\color{blue}\vec{e_i}\cdot\vec{e_j}\endgroup\right) = \begingroup\color{cyan}v^i v^j \endgroup \begingroup\color{blue}g_{ij}\endgroup\\
    &= \begingroup\color{magenta}\tilde{v^i}\tilde{v^j} \endgroup \left(\begingroup\color{red}\tilde{\vec{e_i}}\endgroup \cdot \begingroup\color{red}\tilde{\vec{e_j}}\endgroup\right) = \begingroup\color{magenta}\tilde{v^i} \tilde{v^j} \endgroup \begingroup\color{red}\tilde{g_{ij}}\endgroup
\end{align*}

\begin{subequations}
\begin{empheq}[box=\widefbox]{align}
    \begingroup\color{blue}g_{ij}\endgroup &= \left(\begingroup\color{blue}\vec{e_i}\endgroup\cdot\begingroup\color{blue}\vec{e_j}\endgroup\right)\\
    \begingroup\color{red}\tilde{g_{ij}}\endgroup &= \left(\begingroup\color{red}\tilde{\vec{e_i}}\endgroup\cdot\begingroup\color{red}\tilde{\vec{e_j}}\endgroup\right)
\end{empheq}
\end{subequations}

The metric tensor is actually able to not only get us the length of vectors, but also measure angles.
Let's see how that works, imagine a new basis wrt to an old one, with just one basis vector rotated by an angle $\theta$:

\begin{center}
\begin{tikzpicture}[scale=1.5]
  \draw[step=1cm,gray!30,very thin] (-2,-1) grid (3,3);
  \draw[->,thick] (-2,0) -- (3,0) node[right] {$x$};
  \draw[->,thick] (0,-1) -- (0,3) node[above] {$y$};

  % Old basis (blue)
  \draw[->,very thick,blue] (0,0) -- (2,0) node[below right] {$\vec{e}_1$};
  \draw[->,very thick,blue] (0,0) -- (0,2) node[left] {$\vec{e}_2$};

  % New basis (red)
  \coordinate (O) at (0,0);
  \coordinate (A) at (2,0);
  \coordinate (B) at (1,1.73);

  \draw[->,very thick,red] (O) -- (A) node[above right] {$\tilde{\vec{e}}_1$};
  \draw[->,very thick,red] (O) -- (B) node[left] {$\tilde{\vec{e}}_2$};

  % Angle theta
  \draw[red] (A) pic[
      draw=red,
      ->,
      "$\theta$",
      angle radius=0.6cm,
      angle eccentricity=1.3
  ] {angle = A--O--B};

\end{tikzpicture}

\end{center}

Here, the forward transformation from old to new can be easily found:

\begin{align*}
    \begingroup\color{red}\tilde{\vec{e_1}}\endgroup &= \begingroup\color{blue}\vec{e_1}\endgroup\\
    \begingroup\color{red}\tilde{\vec{e_2}}\endgroup &= \cos(\theta)\begingroup\color{blue}\vec{e_1}\endgroup + \sin(\theta)\begingroup\color{blue}\vec{e_2}\endgroup
\end{align*}

Let's compute the basis dot products in the new basis then:

\begin{align*}
    \begingroup\color{red}\tilde{\vec{e_1}}\endgroup \cdot \begingroup\color{red}\tilde{\vec{e_1}}\endgroup &= \begingroup\color{blue}\vec{e_1}\endgroup \cdot \begingroup\color{blue}\vec{e_1}\endgroup = 1\\
    \begingroup\color{red}\tilde{\vec{e_1}}\endgroup \cdot \begingroup\color{red}\tilde{\vec{e_2}}\endgroup &= \begingroup\color{blue}\vec{e_1}\endgroup \cdot \left(\cos(\theta)\begingroup\color{blue}\vec{e_1}\endgroup + \sin(\theta)\begingroup\color{blue}\vec{e_2}\endgroup\right)\\
    &= \cos(\theta)\left(\begingroup\color{blue}\vec{e_1}\endgroup \cdot \begingroup\color{blue}\vec{e_1}\endgroup\right) + \sin(\theta) \left(\begingroup\color{blue}\vec{e_1}\endgroup \cdot \begingroup\color{blue}\vec{e_2}\endgroup\right)
    = \cos(\theta)\\
    \begingroup\color{red}\tilde{\vec{e_2}}\endgroup \cdot \begingroup\color{red}\tilde{\vec{e_2}}\endgroup 
    &= \left(\cos(\theta)\begingroup\color{blue}\vec{e_1}\endgroup + \sin(\theta)\begingroup\color{blue}\vec{e_2}\endgroup\right) \cdot \left(\cos(\theta)\begingroup\color{blue}\vec{e_1}\endgroup + \sin(\theta)\begingroup\color{blue}\vec{e_2}\endgroup\right)\\
    &= \cos(\theta)^2 \left(\begingroup\color{blue}\vec{e_1}\endgroup \cdot \begingroup\color{blue}\vec{e_1}\endgroup\right) + \sin(\theta)^2 \left(\begingroup\color{blue}\vec{e_2}\endgroup \cdot \begingroup\color{blue}\vec{e_2}\endgroup\right) + \left(2\cos(\theta)\sin(\theta)\right)
    \left(\begingroup\color{blue}\vec{e_1}\endgroup \cdot \begingroup\color{blue}\vec{e_2}\endgroup\right)\\
    &= \cos(\theta)^2 + \sin(\theta)^2 = 1
\end{align*}

So the metric tensors in both basis are:

\begin{align*}
    g_{\color{blue}\vec{e_i}} = \begin{bmatrix}
        1 & 0\\
        0 & 1
    \end{bmatrix} &&
    g_{\color{red}\tilde{\vec{e_i}}} = \begin{bmatrix}
        1 & \cos(\theta)\\
        \cos(\theta) & 1
    \end{bmatrix}
\end{align*}

Allright, now we can imagine having two vectors $\color{violet}\vec{w}$ and $\color{orange}\vec{v}$, for which 
we want to find the angle $\theta$ in between. We can define a new basis vector couple, where the basis vectors are pointing in the same direction as the two
vectors, except for their length be unitary and then we can compute the dot product between two two vectors:

\begin{center}

\begin{tikzpicture}[scale=1.5]

  % Vector w
  \coordinate (W) at (1,1.73);

  % Vector v
  \coordinate (V) at (2,0);

  % New basis (red)
  \coordinate (O) at (0,0);
  \coordinate (A) at (0.7,0);
  \coordinate (B) at (0.33,0.57);


  \draw[->,very thick,violet] (O) -- (W) node[above right] {$\vec{w}$};
  \draw[->,very thick,red] (O) -- (A) node[above right] {$\tilde{\vec{e}}_1$};
  \draw[->,very thick,red] (O) -- (B) node[left] {$\tilde{\vec{e}}_2$};
  \draw[->,very thick,orange] (O) -- (V) node[above right] {$\vec{v}$};

  % Angle theta
  \draw[red] (A) pic[
      draw=red,
      ->,
      "$\theta$",
      angle radius=0.6cm,
      angle eccentricity=1.3
  ] {angle = A--O--B};

\end{tikzpicture}
\end{center}

Since we chose such basis vectors, basically $\color{violet}\vec{w}$ is a scaled $\color{red}\tilde{\vec{e_1}}$ and 
$\color{orange}\vec{v}$ is a scaled $\color{red}\tilde{\vec{e_2}}$ so we can basically write something like:

\begin{align*}
    \left(\begingroup\color{orange}\vec{v}\endgroup \cdot \begingroup\color{violet}\vec{w}\endgroup\right) &= \left(a \begingroup\color{red}\tilde{\vec{e_1}}\endgroup \right) \cdot \left(a \begingroup\color{red}\tilde{\vec{e_2}}\endgroup \right)\\
    &= ab \left(\begingroup\color{red}\tilde{\vec{e_1}}\endgroup \cdot \begingroup\color{red}\tilde{\vec{e_2}}\endgroup\right)\\
    &= ab \cos(\theta)
\end{align*}

Now, since $a$ is just the length of $\color{orange}\vec{v}$ and $b$ the length of $\color{violet}\vec{w}$:

\begin{equation}
   \frac{\left(\begingroup\color{orange}\vec{v}\endgroup \cdot \begingroup\color{violet}\vec{w}\endgroup\right)}
   {\|\begingroup\color{orange}\vec{v}\endgroup\| \|\begingroup\color{violet}\vec{w}\endgroup\|}= \cos(\theta)
\end{equation}

