\section{Motivation: Coordinate Independence in Physics}

Physical laws must not depend on how we label space.

\subsection*{Example: Velocity}
A particle moving in space has a velocity that exists independently of coordinates.
If we rotate our coordinate axes, the particle does not suddenly move differently — only the \emph{numbers} describing its velocity change.

This distinction between:
\begin{itemize}
    \item the geometric object (velocity)
    \item its coordinate representation (components)
\end{itemize}
is the core motivation for tensors.

\subsection*{Key Principle}
\begin{quote}
If a quantity represents something physical or geometric, then changing coordinates must not change the object itself.
\end{quote}

Tensors are defined so this principle is automatically satisfied.
